
\documentclass{article}
\usepackage{graphicx}

\begin{document}
\title{Rest Frame i-band Survey for SNIa z $\sim$ 1.2}
\maketitle

The rest frame $Y$-band survey for the SNIa is constrained by the maximum redshift that can be observed due to the observer frame filter in which the light will be emitted. it is also constrained by 
the limiting magnitude in the YJH bands. 
The aim of this investigation was to look at the contribution of a high-z arm from SNe measured in the $i$-band

\section{Motivation}
Laureijs et al. 2012 present a case for a rest-frame $i$-band SN survey, where they say they will discover ~3000 out to z $\sim$ 1.2.

%\textbf{1. how does this compare to the limiting magnitudes they have mentioned in the report. 
%2. if this is deep drilling fields, does the cadence and survey area + duration allow for these many SNe out to such faint magnitudes. 3000 is realistic for the wide survey, (or the dedicated, Astier-esque, survey)}



The argument was based on the efforts of Freedman et al to get data out to z $\sim$ 0.7 and have a small error budget despite only 35 SNe in their sample. They demonstrate that the scatter is significantly lesser in their $i$ band measurements compare to the $B$-band

\section{Current Design}
The aim of this investigation was to look at the impact of a high-z arm extending out to $z \sim$ 1.2 on decreasing the expanse of the contours for $w_0$-$w_a$. 
We ran 200 simulations for the $z$ range between 1 and 1.2 from an i-band template light curve. using an input cosmology, we convert these into luminosity distances ( after normalising the peak magnitudes.)

\section{Method}
A template light curve was created using the same procedure as for the $YJH$ bands. The mean scatter was higher than at longer wavelengths (as expected from the distribution of $M_i$ for the CSP SNe).
From the template curve, we generate a few hundred realisations with redshifts between 1 and 1.2.

We fit the realisations using SNooPy's i-band template fitter. The final input to the cosmology is the absolute magnitude, redshift and error on the fit. 


\section{Comparison}
In this section, we compare the results for the $w_0$-$w_a$ contours from Union 2.1 distance moduli with the values from the rest frame IR survey. 
The contours are calculated from a $\chi^2$ minimisation

\begin{figure}
\includegraphics[width=.8\textwidth]{/Users/lapguest/cosmo/w/pdf_likelihood/wa_test.pdf}
\caption{$w_a$-$w_0$ contours from the rest frame infrared survey with the high redshift arm measured in the $i$-band}
\end{figure}

\begin{figure}
\includegraphics[width=.8\textwidth]{/Users/lapguest/cosmo/w/pdf_likelihood/wa_test_21.pdf}
\caption{Same as above for Union 2.1}
\end{figure}

We find from the figures that there is a significant improvement in the $w_0$-$w_a$ contours using SNe only from the IR, compared to the Union 2.1 estimates. 

From the Union 2.1 paper, using their CMB+BAO+SNe constraints they obtain a $w_a$ of 0.14 $+$ 0.60 $-$0.76

\section{Complete $i$-band investigation}
In this section, we look at the complete survey being conducted in the rest frame $i$ band, akin to the study of Freedman et al. [2009]. We extend the hubble diagram out to z $\sim$ 1.2, just as in the case mentioned above. In this case, even the lower redshifts are $i$-band peak magnitude values. 

the $w$-$\Omega_{m}$ contours are plotted in figure \ref{fig:i_comp}. There appears to be a significant improvement over the current errors from the JLA and Union 2.1 samples. 

\begin{figure}
\includegraphics[width=.8\textwidth]{/Users/lapguest/cosmo/w/pdf_likelihood/iband_complete.pdf}
\caption{$w$-$\Omega_{m}$ contour for 608 simulated SN events in the $i$-band}
\label{fig:i_comp}
\end{figure}



\end{document}